\documentclass[a4paper,12pt]{article}
\usepackage{amsmath}
\usepackage{pdfpages}
\usepackage[utf8]{inputenc}
\usepackage{hyperref}
\usepackage{listings}
\usepackage{xcolor}
\usepackage{calc}
\usepackage{graphicx}
\usepackage{float}
 
\definecolor{codegreen}{rgb}{0,0.6,0}
\definecolor{codegray}{rgb}{0.5,0.5,0.5}
\definecolor{backcolour}{rgb}{0.95,0.95,0.92}

\lstdefinestyle{mystyle}{
    backgroundcolor=\color{backcolour},   
    commentstyle=\color{codegreen},
    keywordstyle=\color{blue},
    numberstyle=\tiny\color{codegray},
    basicstyle=\ttfamily\footnotesize,
    breakatwhitespace=false,         
    breaklines=true,                 
    captionpos=b,                    
    keepspaces=true,                 
    numbers=left,                    
    numbersep=5pt,                  
    showspaces=false,                
    showstringspaces=false,
    showtabs=false,                  
    tabsize=2
}

\lstset{style=mystyle}

\title{Contol theory Homework \#2 report.\\Variant F.}
\author{Anton Brisilin, BS18-02 Student}
\date{\today}
\begin{document}
\maketitle
\section{Task 1.}
\subsection*{A. Calculate total transfer function}
    Calculate total Transfer Function of the system:
    \begin{center}
        \includegraphics[width=\linewidth/2]{../Task1/ToReport/T1Step0.pdf}
    \end{center}
    As a first step I will simplify feedback chain:
    \begin{center}
        \includegraphics[width=\linewidth/2]{../Task1/ToReport/T1Step1.pdf}
    \end{center}
    And then our system can be represented as one block:
    \begin{center}
        \includegraphics[width=0.3\linewidth]{../Task1/ToReport/T1Step2.pdf}
    \end{center}
    In variant F I have:
    \begin{equation*}
    W_1 = \frac{s+3}{s+1}, W_2 = \frac{1}{s+2}, 
    W_3 = \frac{1}{s+0.1}, W_4 = \frac{1}{2s+3}
    \end{equation*}
    Substituting $W_1, ... ,W_4$ to the system block:
    \footnote{This answer was get with Matlab script simplify\_fraction.m in Task1 folder}
    \begin{equation*}
    W_1 * W_3 * \frac{W_2}{1+W2*W4} = 
    \frac{20s^2 + 90s + 90}{20s^4 + 92s^3 + 149s^2 + 84s + 7}
    \end{equation*}
    Hence, our total Transfer Function is 
    \begin{equation}
        W = \frac{20s^2 + 90s + 90}{20s^4 + 92s^3 + 149s^2 + 84s + 7}
    \end{equation}
    \subsection*{B. Build initial and simplified systems and analyse responses}
    \begin{center}
        \includegraphics[width=\linewidth]{../Task1/ToReport/T1Model.pdf}
        Simulink model for response analysis. For analysing different responses,
        one need to put signal source to input instead of the clock.
    \end{center}
    Step, Impulse, and Frequency responses plots. On each plot blue line is 
    system input, red one is the initial system output, and the orange one is
    system output after simplification:\\
    \begin{figure}[H]
        \centering
        \includegraphics[width=0.35\linewidth]{../Task1/ToReport/T1StepResponse.pdf}        
        \includegraphics[width=0.35\linewidth]{../Task1/ToReport/T1ImpulseResponse.pdf}\\
        \caption{Step response (left) and Impulse response (right) plots}
        \label{fig1}
    \end{figure}
    
    \begin{figure}[H]
        \centering
        \includegraphics[width=0.35\linewidth]{../Task1/ToReport/T1FreqResponse.pdf}
        \caption{Frequency response plot}
        \label{fig2}
    \end{figure}

    Note, that plots look like there is only one output, but it is not true -- 
    they are just very close to each other.

    \subsection*{C. Bode and Pole-Zero map plots}
    As a input signal, for which I will generate the plots I have chosen 
    a Frequency response (sine function).
    \begin{center}
        \includegraphics[width=\linewidth]{../Task1/ToReport/T1Bode.pdf}\\
        \includegraphics[width=\linewidth]{../Task1/ToReport/T1PoleZero.pdf}        
    \end{center}
    According to Matlab \texttt{pzmap()} function, our Transfer Function (1) has following poles 
    and zeroes:
    \begin{equation*}
    poles =
    \begin{bmatrix}
      -1.7500 + 0.6614i\\
      -1.7500 - 0.6614i\\
      -1.0000 + 0.0000i\\
      -0.1000 + 0.0000i
    \end{bmatrix} 
    zeroes =
    \begin{bmatrix}
        -3.0000\\
        -1.5000
    \end{bmatrix}
    \end{equation*}
    As it can be seen, all our poles have negative real part, and that says us 
    that the system is stable.
\newpage
\section{Task 2.}
    Find total Transfer Function of the system:
    \begin{center}
        \includegraphics[scale=0.5]{../Task2/ToReport/System.pdf}
    \end{center}
    Firstly, let's consider case when $f(t)=0$. In this case our Transfer
    Function is $\frac{W}{1+W}$. Then we'll consider case, when $g(t)=0$, 
    and our Transfer Function will be $M \times \frac{1}{1+W}$.
    Hence, total transfer function for the system in general case is
    \begin{equation*}
        output = \frac{W}{1+W} \times g(t) + \frac{M}{1+W} \times f(t)
    \end{equation*}  
    In variant F I have:
    \begin{equation*}
    W(s) = \frac{s+1}{s^2+3s+2}, 
    M(s) = \frac{1}{s+3}
    \end{equation*}
    After simplification of fractions with Matlab, we get 
    \begin{equation*}
        output = \frac{1}{s+3} \times g(t) + \frac{s+1}{(s+3)^2 \times f(t)}
    \end{equation*}

\section{Task 3.}    
    Find transfer function of the system.\\
    In variant F I have:
    \begin{equation*}
        A = \begin{bmatrix}
             1 & 0\\
             2 & 1
        \end{bmatrix},
        B = \begin{bmatrix}
            2 \\ 2
        \end{bmatrix},
        C = \begin{bmatrix}
            -1 & 4
        \end{bmatrix},
        D = \begin{bmatrix}
            2
        \end{bmatrix}
    \end{equation*}
    Using Matlab we can obtain required Transfer Function with \texttt{ss2tf()}:
    \lstinputlisting[language=matlab]{../Task3/ss_to_tf.m}
    Hence, our Transfer Function is
    \begin{equation*}
        H(s) = \frac{2s^2 + 2s + 12}{s^2-2s+1}
    \end{equation*} 
    Matlab's \texttt{ss2tf()} uses the same formula as we used on labs to 
    transform state-space model to Transfer Function: $C*(sI-A)^{-1}*B+D$, 
    where $A$, $B$, $C$, and $D$ are the matrices of state-space model of the 
    system. The other way to solve it - use the Matlab's representation of 
    above formula, that is \texttt{C*inv(s*eye(2)-A)*B+D}, which will return 
    vector of transfer fucntions. But to use \texttt{ss2tf()} for me is shorter 
    and more convenient way.\\
\section{Task 4.}
    Find transfer function of the system.\\
    In variant F I have:
    \begin{equation*}
        A = \begin{bmatrix}
             2 & 1\\
             -3 & 1
        \end{bmatrix},
        B = \begin{bmatrix}
            -1 & 5 \\ 
            3 & 1
        \end{bmatrix},
        C = \begin{bmatrix}
            -2 & 0
        \end{bmatrix},
        D = \begin{bmatrix}
            2 & 4
        \end{bmatrix}
    \end{equation*}
    Using Matlab we can obtain required Transfer Function with \texttt{ss2tf()}:
    \lstinputlisting[language=matlab]{../Task4/ss_to_tf.m}
    Hence, our total Transfer Function is:
    \begin{equation*}
        H(s) = \frac{2s^2 - 4s + 2}{s^2 - 3 s + 5}  u_1 + 
        \frac{4 s^2 - 22 s + 28}{s^2 - 3 s + 5} u_2
    \end{equation*}
\newpage
\section{Task 5.}
    Given system, variant F:
    \begin{center}
        \includegraphics[width=\linewidth]{../Task5/T5Source.pdf}
    \end{center}
    \subsection*{A. Assuming that x is 0:}
    \begin{center}
        \includegraphics[width=0.8\linewidth]{../Task5/T5Step1.pdf}
    \end{center}
    Simplifying series connection of $W6, W2, W4$ blocks:
    \begin{center}
        \includegraphics[width=0.8\linewidth]{../Task5/T5Step2.pdf}
    \end{center}
    Simplifying series connection of $W6*W2*W4$ and $W7$ blocks, and then series 
    connection of resulting block with block $W8$:
    \begin{center}
        \includegraphics[width=\linewidth/2]{../Task5/T5Step3.pdf}
    \end{center}
    \subsection*{B. Assuming that f is 0:}
    \begin{center}
        \includegraphics[width=\linewidth]{../Task5/T5Step4.pdf}
    \end{center}
    Let's examine chains $W1, W7$, and $W1, W6, W5, W2, W4$ as separate 
    subchains:
    \begin{center}
        \includegraphics[width=\linewidth]{../Task5/T5Step5.pdf}
    \end{center}
    Collapsing chains $W1,W6,W5$ and $W1, W7$  to blocks $W1 * W6 + W5$, and 
    $W1 * W7$ respectively: 
    \begin{center}
        \includegraphics[width=\linewidth]{../Task5/T5Step6.pdf}
    \end{center}
    Simplifying series connection of $W1 * W6 + W5, W2, W4$ blocks:
    \begin{center}
        \includegraphics[width=\linewidth]{../Task5/T5Step7.pdf}
    \end{center}
    Simplifying parallel connection:
    \begin{center}
        \includegraphics[width=0.8\linewidth]{../Task5/T5Step8.pdf}
    \end{center}
    Simplifying series connection with block $W8$:
    \begin{center}
        \includegraphics[width=0.8\linewidth]{../Task5/T5Step9.pdf}
    \end{center}
    \subsection*{C. Total Transfer Function}
    After all steps, our total Transfer Function will look like:
    \begin{equation*}
        output = W_8(((W_1 W_6 + W_5) W_2 W_4 + W_3 + W_1 W_7) \times x + 
        (W_6 W_2 W_4 + W_7) \times f)
    \end{equation*}


\section{Used software.}
\begin{itemize}
    \item Python 3.8.1
    \item Matlab R2018b 9.5.0
    \item draw.io
\end{itemize}
All software was run under Manjaro Linux with 5.4.18-rt kernel
\end{document}