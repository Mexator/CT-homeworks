\documentclass[a4paper,12pt]{article}
\usepackage{amsmath}
\usepackage{pdfpages}
\usepackage[utf8]{inputenc}
\usepackage{hyperref}
\usepackage{listings}
\usepackage{xcolor}
\usepackage{fancyhdr}
\definecolor{codegreen}{rgb}{0,0.6,0}
\definecolor{codegray}{rgb}{0.5,0.5,0.5}
\definecolor{backcolour}{rgb}{0.95,0.95,0.92}

\lstdefinestyle{mystyle}{
    backgroundcolor=\color{backcolour},   
    commentstyle=\color{codegreen},
    keywordstyle=\color{blue},
    numberstyle=\tiny\color{codegray},
    basicstyle=\ttfamily\footnotesize,
    breakatwhitespace=false,         
    breaklines=true,                 
    captionpos=b,                    
    keepspaces=true,                 
    numbers=left,                    
    numbersep=5pt,                  
    showspaces=false,                
    showstringspaces=false,
    showtabs=false,                  
    tabsize=2
}

\lstset{style=mystyle}

\pagestyle{fancy}
\fancyhf{}
\lhead{Contol theory Homework \#5 report}
\rhead{Anton Brisilin, BS18-02 Student}
\fancyfoot[R]{\today}
\fancyfoot[C]{\thepage}
\renewcommand{\footrulewidth}{1pt}
\renewcommand{\headrulewidth}{1pt}

\begin{document}
\section{Task 1}
My name is \textit{Anton Brisilin}, and my email is 
\textit{a.brisilin@innopolis.university}, so my generated variant is \textbf{F}.

\section{Task 2}
First, lets obtain our new linearized system (with new coefficients).\\
From the previous assignment I obtained such nonlinear dynamics equation 
for the pendulum on a cart system:
\begin{equation*}
    \begin{bmatrix}
        \dot x\\
        \dot \theta\\
        \ddot x\\
        \ddot \theta    
    \end{bmatrix}
    =
    \begin{bmatrix}
        \dot x\\
        \dot \theta\\
        \frac
        {- mlsin(\theta)\dot \theta^2 + mgsin(\theta)cos(\theta)}
        {(M+m) - mcos^2(\theta)}\\
        \frac
        {(M+m)gsin(\theta) - mlcos(\theta)sin(\theta)\dot \theta^2}
        {l((M+m) - mcos^2(\theta))}
    \end{bmatrix}
    +
    \begin{bmatrix}
        0\\0\\
        \frac
        {1}
        {(M+m) - mcos^2(\theta)}\\
        \frac
        {cos(\theta)}
        {(M+m) - mcos^2(\theta)}
    \end{bmatrix}
    u
\end{equation*}
Just like in the previous assignment, I should linearize it by computing derivatives.
I will not pay too much attention to it, the script for computing derivatives is 
located as \texttt{Task2/diff.m}
\begin{equation}\label{linearized}
    \begin{bmatrix}
        \delta \dot x\\
        \delta \dot \theta\\
        \delta \dot x\\
        \delta \ddot \theta\\
    \end{bmatrix}
    =
    \begin{bmatrix}
        0 & 0 & 1 & 0 \\
        0 & 0 & 0 & 1 \\
        0 & 0.7796 & 0 & 0\\
        0 & 30.256 & 0 & 0\\
    \end{bmatrix}
    +
    \begin{bmatrix}
        0\\
        0\\
        0.066\\
        0.066\\
    \end{bmatrix}
    u
\end{equation}
\subsection{Prove that it is possible to design state observer of the linearized
system}
In order to make possible existence of state observer, the observed system should
be fully observable.\\
To prove observability of the obtained system, I will use Principle of Duality.
It says, that if we have system
\begin{equation*}    
    \begin{cases}
        \dot x = Ax + Bu\\
        y = Cx + Du
    \end{cases}
\end{equation*}
then we can determine whether it is observable or not by considering controllability 
of its dual system, that is:
\begin{equation*}    
    \begin{cases}
        \dot z = A^*z + C^*k\\
        t = B^*z
    \end{cases}
\end{equation*}
where $A^*, B^*, C^*, D^*$ are conjugate transposes of matrices $A,B,C$ and $D$ respectively.
In our case $A$ and $B$ are same as in (\ref{linearized}). Matrix $C$ looks like this:
\begin{equation*}
    C=
    \begin{bmatrix}
        1 & 0 & 0 & 0\\
        0 & 1 & 0 & 0\\
        0 & 0 & 0 & 0\\
        0 & 0 & 0 & 0\\
    \end{bmatrix}
\end{equation*}
Thus, our conjugate transpose matrices are
\begin{equation*}
    \begin{cases}
        A^*= 
        \begin{bmatrix}
            0 & 0 & 0 & 0\\
            0 & 0 & 0.7796 & 30.256\\
            1 & 0 & 0 & 0\\
            0 & 1 & 0 & 0\\
        \end{bmatrix}\\
        B^*=
        \begin{bmatrix}
            0 & 0 &  0.066 & 0.066 
        \end{bmatrix}\\
        C^*=
        \begin{bmatrix}
            1 & 0 & 0 & 0\\
            0 & 1 & 0 & 0\\
            0 & 0 & 0 & 0\\
            0 & 0 & 0 & 0\\
        \end{bmatrix}\\
    \end{cases}
\end{equation*}
According to Ogata, "necessary and sufficient condition for complete observability [of my first system]
is that the rank of the $n\times nm$ matrix 
\begin{center}
    $[C^* | A^*C^* | \dots | (A^*)^{(n-1)}C^*]$
\end{center}
be $n$". Fortunately, Matlab has built-in function to check for controllability 
building controllability matrix. By checking its rank (that is 4 - the number of 
dimensions of $A$) with script in \texttt{Task2/contr.m}, we cane make conclusion,
that it is observable.

\subsection{For open-loop observation: is the error dynamics stable?}
Open-loop observation: the idea is that if we know exact model, and initial 
conditions, we can estimate state of the system even not considering input and 
output.\\
Let's look at our linearized system, and check it for stability by looking at 
eigenvalues of $A$.
\begin{equation*}
    eig(A) = 
    \begin{bmatrix}
        0. & 0. & 5.50054543 & -5.50054543
    \end{bmatrix}
\end{equation*}  
As we can see, system is not stable, hence error dynamics (\ref{error_dyn}) is not 
stable,
\begin{equation} \label{error_dyn}
    \dot \epsilon = A\epsilon,
\end{equation} 
too. We can not say that error eventually will go to zero, because it 
will not, if observer used in the model considers initial conditions different 
from actual ones.
\subsection{Design Luenberger observer}

\section{Used software}
\begin{itemize}
    \item Python 3.8.1
    \item Matlab R2018b 9.5.0
    \item draw.io
\end{itemize}
All software was run under Manjaro Linux with 5.4.18-rt kernel
\end{document}