\documentclass[a4paper,12pt]{article}
\usepackage{amsmath}
\usepackage{pdfpages}
\usepackage[utf8]{inputenc}
\usepackage{hyperref}
\usepackage{listings}
\usepackage{xcolor}
\usepackage{fancyhdr}
\definecolor{codegreen}{rgb}{0,0.6,0}
\definecolor{codegray}{rgb}{0.5,0.5,0.5}
\definecolor{backcolour}{rgb}{0.95,0.95,0.92}

\lstdefinestyle{mystyle}{
    backgroundcolor=\color{backcolour},   
    commentstyle=\color{codegreen},
    keywordstyle=\color{blue},
    numberstyle=\tiny\color{codegray},
    basicstyle=\ttfamily\footnotesize,
    breakatwhitespace=false,         
    breaklines=true,                 
    captionpos=b,                    
    keepspaces=true,                 
    numbers=left,                    
    numbersep=5pt,                  
    showspaces=false,                
    showstringspaces=false,
    showtabs=false,                  
    tabsize=2
}

\lstset{style=mystyle}

\pagestyle{fancy}
\fancyhf{}
\lhead{Contol theory Homework \#4 report}
\rhead{Anton Brisilin, BS18-02 Student}
\fancyfoot[R]{\today}
\fancyfoot[C]{\thepage}
\renewcommand{\footrulewidth}{1pt}
\renewcommand{\headrulewidth}{1pt}

\begin{document}
\section{Task 1}
My name is \textit{Anton Brisilin}, and my email is 
\textit{a.brisilin@innopolis.university}, so my generated variant is \textbf{F}.
\section{Task 2}
    \subsection{Write equations of motion of the system in manipulator form}
    Source system of equations:
    \begin{equation}\label{source_eq}
        \begin{cases}
            (M+m)\ddot{x}-mlcos(\theta)\ddot{\theta}+mlsin(\theta)\dot\theta^2=F\\
            -cos(\theta)\ddot{x}+l\ddot{\theta}-gsin(\theta)=0
        \end{cases}
    \end{equation}
    Manipulator form:
    \begin{equation}
        \begin{cases}\label{desired_form}
            M(q)\ddot{q}+n(q,\dot{q})=Bu\\
            u=F\\
            q=\begin{bmatrix}x \\ \theta\end{bmatrix}
        \end{cases}
    \end{equation}
    My goal is to transform (\ref{source_eq}) to the form of (\ref{desired_form}). 
    To do it I can transform first equation of (\ref{source_eq}) to the form of
    \begin{equation}\label{first_part}
        \begin{bmatrix}
            M+m & -mlcos(\theta)
        \end{bmatrix}
        \begin{bmatrix}
            \ddot x \\ 
            \ddot\theta
        \end{bmatrix}
         + (mlsin(\theta)\dot\theta^2) = [1]u
    \end{equation}
    This already is motion equation in standart (manipulator) form. But for this
    equation to describe motion of our system it should also consider second
    equation of system (\ref{source_eq}). Therefore, our equation (\ref{first_part})
    becomes
    \begin{equation}
        \begin{bmatrix}
            M+m & -mlcos(\theta) \\
            -cos(\theta) & l
        \end{bmatrix}
        \begin{bmatrix}
            \ddot x \\ 
            \ddot\theta
        \end{bmatrix}
        +
        \begin{bmatrix}
            mlsin(\theta)\dot\theta^2\\
            -gsin\theta
        \end{bmatrix}
        = 
        \begin{bmatrix}
            1\\
            0
        \end{bmatrix}u
    \end{equation}
    And with substitution of numbers from variant \textbf{F}:
    \begin{equation}
        \begin{bmatrix}
            11.6 & -7.298 * cos(\theta) \\
            -cos(\theta) & 0.89
        \end{bmatrix}
        \begin{bmatrix}
            \ddot x \\ 
            \ddot\theta
        \end{bmatrix}
        +
        \begin{bmatrix}
            7.298*sin(\theta)\dot\theta^2\\
            -9.81*sin\theta
        \end{bmatrix}
        = 
        \begin{bmatrix}
            1\\
            0
        \end{bmatrix}u
    \end{equation}
    \subsection{Write dynamics of the system in control affine nonlinear form}
    Control affine nonlinear form:
    \begin{equation}\label{desired_form2}
        \begin{cases}
            \dot z = f(z) + g(z)u\\
            u=F\\
            z = 
            \begin{bmatrix}
                x&
                \theta&
                \dot x&
                \dot \theta    
            \end{bmatrix}^T
        \end{cases}
    \end{equation}
    It's pretty obvious that one can easily express $\dot x$ and $\dot \theta$ 
    through $z$:
    \begin{equation}
        \begin{bmatrix}
            \dot x\\
            \dot \theta
        \end{bmatrix}
        =
        \begin{bmatrix}
            \dot x\\
            \dot \theta    
        \end{bmatrix}
        +
        \begin{bmatrix}
            0\\
            0\\    
        \end{bmatrix}F
    \end{equation}
    So the goal is to express $\ddot x$ and $\ddot \theta$ through $x, \dot x, 
    \theta$ and $\dot \theta$. From the second equation of system (\ref{source_eq}):
    \begin{equation}\label{x_ddot}
        \ddot x = \frac{l\ddot\theta - gsin(\theta)}{cos(\theta)}
    \end{equation}
    Substituting (\ref{x_ddot}) to the first equation of (\ref{source_eq}), we can 
    express $\ddot{\theta}$:
    \begin{gather*}
        \ddot{\theta}=
        \frac
        {(M+m)\ddot{x}+mlsin(\theta)\dot \theta^2-F}
        {mlcos(\theta)}
        \\
        \ddot{\theta}=
        \frac
        {(M+m) (l\ddot \theta - gsin(\theta))}
        {mlcos^2(\theta)}
        +
        tg(\theta) \dot \theta^2
        - 
        \frac
        {F}
        {mlcos(\theta)}
        \\
        \ddot{\theta}=
        \frac
        {(M+m)l\ddot\theta}
        {mlcos^2(\theta)}
        - 
        \frac
        {(M+m)gsin(\theta)}
        {mlcos^2(\theta)}
        +
        tg(\theta) \dot \theta^2
        - 
        \frac
        {F}
        {mlcos(\theta)}
        \\
        \ddot{\theta}
        \left(
            \frac
            {mlcos^2(\theta)-(M+m)l}
            {mlcos^2(\theta)}
        \right)=
        - 
        \frac
        {(M+m)gsin(\theta)}
        {mlcos^2(\theta)}
        +
        tg(\theta)\dot \theta^2 
        -
        \frac
        {F}
        {mlcos(\theta)}
        \\
        \ddot\theta =
        \frac
        {-  \frac
            {(M+m)gsin(\theta)}
            {mlcos^2(\theta)}
            +
            tg(\theta)\dot \theta^2
            - 
            \frac
            {F}
            {mlcos(\theta)}
        }
        {
            \frac
            {mlcos^2(\theta)-(M+m)l}
            {mlcos^2(\theta)}
        }
        \\
        \ddot\theta = 
        \frac
        {-(M+m)gsin(\theta) + mlcos(\theta)sin(\theta)\dot\theta^2 - Fcos(\theta)}
        {mlcos^2(\theta)-(M+m)l}
        \\
    \end{gather*}
    \begin{equation}\label{theta_ddot}
        \ddot\theta = 
        \frac
        {(M+m)gsin(\theta) - mlcos(\theta)sin(\theta)\dot \theta^2}
        {l((M+m) - mcos^2(\theta))}
        +
        \frac
        {Fcos(\theta)}
        {l((M+m) - mcos^2(\theta))}
    \end{equation}
    Now, substituting (\ref{theta_ddot}) to (\ref{x_ddot}):
    \begin{gather*}
        \ddot x = 
        \frac
        {(M+m)gtg(\theta) - mlsin(\theta)\dot \theta^2}
        {(M+m) - mcos^2(\theta)}
        +
        \frac
        {F}
        {(M+m) - mcos^2(\theta)}
        -
        gtg(\theta)
    \end{gather*}
    \begin{equation}
        \ddot x = 
        \frac
        {- mlsin(\theta)\dot \theta^2 + mgsin(\theta)cos(\theta)}
        {(M+m) - mcos^2(\theta)}
        +
        \frac
        {F}
        {(M+m) - mcos^2(\theta)}
    \end{equation}
    Hence, our system (\ref{source_eq}) in control affine nonlinear form can be
    written as:
    \begin{equation}
        \begin{bmatrix}
            \dot x\\
            \dot \theta\\
            \ddot x\\
            \ddot \theta    
        \end{bmatrix}
        =
        \begin{bmatrix}
            \dot x\\
            \dot \theta\\
            \frac
            {- mlsin(\theta)\dot \theta^2 + mgsin(\theta)cos(\theta)}
            {(M+m) - mcos^2(\theta)}\\
            \frac
            {(M+m)gsin(\theta) - mlcos(\theta)sin(\theta)\dot \theta^2}
            {l((M+m) - mcos^2(\theta))}
        \end{bmatrix}
        +
        \begin{bmatrix}
            0\\0\\
            \frac
            {1}
            {(M+m) - mcos^2(\theta)}\\
            \frac
            {cos(\theta)}
            {(M+m) - mcos^2(\theta)}
        \end{bmatrix}
        u
    \end{equation}
    And with substitution of numbers from variant \textbf{F}:
    % Everything becomes even more ugly
    \begin{equation}
        \begin{bmatrix}
            \dot x\\
            \dot \theta\\
            \ddot x\\
            \ddot \theta    
        \end{bmatrix}
        =
        \begin{bmatrix}
            \dot x\\
            \dot \theta\\
            \frac
            {- 7.298sin(\theta)\dot \theta^2 + 80.442sin(\theta)cos(\theta)}
            {11.6 - 8.2cos^2(\theta)}\\
            \frac
            {113.796sin(\theta) - 7.298cos(\theta)sin(\theta)\dot \theta^2}
            {10.324 - 7.298cos^2(\theta)}
        \end{bmatrix}
        +
        \begin{bmatrix}
            0\\0\\
            \frac{1}
            {11.6 - 8.2cos^2(\theta)}\\
            \frac{cos(\theta)}
            {11.6 - 8.2cos^2(\theta)}\\
        \end{bmatrix}
        u
    \end{equation}
    \subsection{Linearize nonlinear dynamics of the systems around equilibrium point
    $\vec{z} = [0\ 0\ 0\ 0]^T$}
    General form of linearized equation:
    \begin{equation}
        \delta \dot z = A \delta z + B \delta u 
    \end{equation}
    Let's begin with checking, whether $\vec{z} = [0\ 0\ 0\ 0]^T$ is really an 
    equilibrium point:
    \begin{equation}
        \begin{cases}
            \dot x = 0\\
            \dot \theta = 0\\
            \ddot x = 0 +  
            \frac{1}
            {11.6 - 8.2cos^2(\theta)}
            u_e = 0\\
            \ddot \theta = 0 +  
            \frac{cos(\theta)}
            {11.6 - 8.2cos^2(\theta)}u_e =0
        \end{cases}
    \end{equation}
    It's seen that last two equations are true, iff $u_e=0$.
    If our system is in form
    \begin{equation}
        \begin{bmatrix}
            \dot x_1\\
            \dot x_2\\
            \dot x_3\\
            \dot x_4\\
        \end{bmatrix}
        =
        \begin{bmatrix}
            f_1(z,u)\\
            f_2(z,u)\\
            f_3(z,u)\\
            f_4(z,u)\\
        \end{bmatrix}
    \end{equation}
    Then, to linearize it, we should found such matrices $A$ and $B$, that
    \begin{equation}
        \begin{cases}
            A=
            \begin{bmatrix}
                \frac
                {\partial f_1}
                {\partial z_1}|_{z_0,u_0} &
                \frac
                {\partial f_1}
                {\partial z_2}|_{z_0,u_0} &
                \frac
                {\partial f_1}
                {\partial z_3}|_{z_0,u_0} &
                \frac
                {\partial f_1}
                {\partial z_4}|_{z_0,u_0} \\
                \frac
                {\partial f_2}
                {\partial z_1}|_{z_0,u_0} &
                \frac
                {\partial f_2}
                {\partial z_2}|_{z_0,u_0} &
                \frac
                {\partial f_2}
                {\partial z_3}|_{z_0,u_0} &
                \frac
                {\partial f_2}
                {\partial z_4}|_{z_0,u_0} \\
                \frac
                {\partial f_3}
                {\partial z_1}|_{z_0,u_0} &
                \frac
                {\partial f_3}
                {\partial z_2}|_{z_0,u_0} &
                \frac
                {\partial f_3}
                {\partial z_3}|_{z_0,u_0} &
                \frac
                {\partial f_3}
                {\partial z_4}|_{z_0,u_0} \\
                \frac
                {\partial f_4}
                {\partial z_1}|_{z_0,u_0} &
                \frac
                {\partial f_4}
                {\partial z_2}|_{z_0,u_0} &
                \frac
                {\partial f_4}
                {\partial z_3}|_{z_0,u_0} &
                \frac
                {\partial f_4}
                {\partial z_4}|_{z_0,u_0} \\
            \end{bmatrix}\\
            B=
            \begin{bmatrix}
                \frac
                {\partial f_1}
                {\partial u}|_{z_0,u_0} \\
                \frac
                {\partial f_2}
                {\partial u}|_{z_0,u_0} \\
                \frac
                {\partial f_3}
                {\partial u}|_{z_0,u_0} \\
                \frac
                {\partial f_4}
                {\partial u}|_{z_0,u_0} \\
            \end{bmatrix}
        \end{cases}
    \end{equation}
    In our case, functions $f_1, ... f_4$ can be taken from previous part of the
    task. Now let's compute the matrices, considering
    \begin{equation}
        \begin{cases}
        z = 
        \begin{bmatrix}
            z_1\\
            z_2\\
            z_3\\
            z_4\\
        \end{bmatrix} = 
        \begin{bmatrix}
            x\\
            \theta\\
            \dot x\\
            \dot \theta    
        \end{bmatrix}\\
        f_1 = z_3\\
        f_2 = z_4\\
        f_3 = \frac
        {- 7.298sin(z_2)z_4^2 + 80.442sin(z_2)cos(z_2)}
        {11.6 - 8.2cos^2(z_2)}
        +
        \frac{u}
        {11.6 - 8.2cos^2(z_2)}
        \\
        f_4 = \frac
        {113.796sin(z_2) - 7.298cos(z_2)sin(z_2)z_4^2}
        {10.324 - 7.298cos^2(z_2)}
        +
        \frac{cos(z_2)u}
        {11.6 - 8.2cos^2(z_2)}\\
        \end{cases}
    \end{equation}
    \begin{equation}
        \begin{cases}
            A=
            \begin{bmatrix}
                0 & 0 & 1 & 0 \\
                0 & 0 & 0 & 1 \\
                0 & 23.66 & 0 & 0\\
                0 & 37.6  & 0 & 0\\
            \end{bmatrix}\\
            B=
            \begin{bmatrix}
                0 \\
                0 \\
                0.086 \\
                0.294 \\
            \end{bmatrix}
        \end{cases}
    \end{equation}
    All the derivatives were computed with Wolfram Alpha and my calculator.
    Inputs for Wolfram differentiation can be found in \texttt{Task2/wolfram\_inputs.txt}
\section{Used software.}
\begin{itemize}
    \item Python 3.8.1
    \item Matlab R2018b 9.5.0
    \item draw.io
\end{itemize}
All software was run under Manjaro Linux with 5.4.18-rt kernel
\end{document}